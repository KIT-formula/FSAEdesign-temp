%%%%%%%%%%%%%%%%%%%%%%%%%%%
\section{車両開発方針}
%%%%%%%%%%%%%%%%%%%%%%%%%%%
車両開発の目標であるシングルナンバーを獲得するため,動的競技のうち最も総合順位との相関のあるエンデュランスに着目した.過去3年の大会結果から,総合順位一桁のチームはエンデュランスにおいても高順位であることが分かった.そこで昨年度の結果から,エンデュランス総合タイムが1320 sec(平均タイム66.0 sec/lap)で総合順位が8位以上を獲得できることが分かったため,これを車両開発目標に設定し開発を行った.

%%%%%%%%%%%%%%%%%%%%%%%%%%%
\subsection{ベンチマーク分析}
%%%%%%%%%%%%%%%%%%%%%%%%%%%
前項にて設定したエンデュランス総合タイムで走行していた2017年度大会の他大学車両と,KS-14との間でどういった区間でタイム差が生じているかをベンチマークとして,昨年度大会の動画から検証を行った.動画から区間A,Bを設定し(区間A:高速状態での加速性能,旋回性能が重視される区間,区間B:低速からの加速性能,応答性能が重視される区間)分析を行ったところ,区間Aでは平均して$\Delta$1.557 sec,区間Bでは$\Delta$2.379 secと,KS-14とベンチマークとの間に差があることが分かった.そこでパワートレインでは主にエンジン低回転領域からの加速性能,サスペンションでは旋回性能に加え応答性能を重視して開発を行った.またフレームやエアロデバイスといったボディではサスペンションの性能要求を満たすこと,コクピットにおいては従来のDrivabilityに加え,今まで重要視していなかったドライバーの快適性(以下Comfortability)を追求することとした.またサスペンションからの性能要求を満たすため,各パーツにおいて軽量化の検討を行った.
