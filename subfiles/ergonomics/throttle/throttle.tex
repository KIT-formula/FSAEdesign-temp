%%%%%%%%%%%%%%%%%%%%%%%%%%%
\subsection{電子スロットル}
%%%%%%%%%%%%%%%%%%%%%%%%%%%
%%問題点
従来の車両はシフトチェンジ時にトランスミッションによって生じる車両のバタつきや過度のエンジンブレーキによるスリップが問題となっていた.

%%改善点
そこで今年度は電子スロットルを採用し,アクセルワークのアシストを行うこととした.まず,従来通りのペダルの応答性に近づけるため人間の脳の反応速度0.1 secを目標値とし,ペダル入力から0.1 sec以内にモータへ出力がなされるシステムを構築した.シフトダウン時にブリッピング介入を施す事で回転数を調節し,車両挙動の安定化に努めた.また,誤動作を防止するために配線をツイスト状にすることやメインハーネスとモータへの配線に距離を設けることでノイズ対策にも努めた.ドライバーの操作をより滑らかに実現するためにペダルのストローク角を1024段階のデジタルデータで扱い,従来同然の動きを追求した.さらに,ドライバーの意思で介入処理の有無を操作できるようにしドライバーニーズの多様化への対応も行なった.
