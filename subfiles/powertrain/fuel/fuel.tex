%%%%%%%%%%%%%%%%%%%%%%%%%%%
\subsection{燃料}
%%%%%%%%%%%%%%%%%%%%%%%%%%%
’18年度のチーム目標を達成するためにはエンデュランス完走が前提であり,またパワートレインの目標を達成するためには燃料系の総重量は6000gを切る必要があると考えた.そこで,燃料システムの目標を「安定した燃料供給と軽量化」とした.なお,この総重量は“’18 年度目標マシン重量“に“マシン全体に対する燃料系の重量割合“を掛けて求めたものである.

これを踏まえて,製作目標は「エンデュランス完走直後の燃料残量の状態でコーナーに侵入した際でもエア噛みが発生しないタンク」とし,解析ソフトANSYSから得られた解析結果からこれに見合うタンクの形状および容量を決定した.その際の解析条件を,“タンク内に0.7 Lの燃料残量がある状態で最大横G 1.8 G及び最大縦G 0.5 Gが発生する”とした.なお,エンデュランス完走直後の燃料残量0.7 Lは,タンクの容量5.0 L(’15年度大会のタンク容量が5 Lだったことからこれを指標とした)から燃料の使用量4.3 L(気温$20 \ {}^\circ\mathrm{C}$時の試走会で測定した燃費により決定)を引いたものである. 

上記の解析条件よりタンクの形状を決定したが,形状の工夫だけでは旋回時に発生する液体の偏りを抑えることが出来ず,一時的ではあるがエア噛みが発生する瞬間があった.そこで,液体の偏り及び挙動を抑えるために’18年度から構造体内にバッフルプレートを設けることにした(Fig.\ref{fig:fuel1}).その結果,燃料残量0.7 L時に旋回Gが発生した際の液体の挙動が小さくなり,エア噛みの発生を抑えることが出来た(Fig.\ref{fig:fuel2}).これにより,燃料タンクの容量を17年度の5.3 Lから5.0 Lへの変更が可能になり,ガソリンを含めたのタンクの重量を311 g減少することに成功した.また,’17年度のタンクに使用したネックパイプの素材がA5052の$\phi$45 mm,t5 mmであったため,’18年度は同素材の$\phi$41 mm,t2 mmのものに変更することで330 gの軽量化に成功した.さらに,タンクに使用するプレートに曲げ加工を施すことで,タンクの1面につきアルミ板を1枚ずつ切り出すよりも溶接長が647 mm短くなり,179 g(溶接長1 mmにつき0.277 g)の軽量化に成功した.

したがって,’17年度と比較すると全体で990 g(実測値)の軽量化に成功し,’18年度燃料系の総重量はガソリン満タン時で5500 gとなり目標重量を達成した.
