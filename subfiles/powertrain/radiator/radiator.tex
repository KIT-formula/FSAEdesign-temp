%%%%%%%%%%%%%%%%%%%%%%%%%%%
\subsection{冷却}
%%%%%%%%%%%%%%%%%%%%%%%%%%%
Fig.\ref{fig:radiator1}にKS-14とKS-15のラジエータ配置を示す.KS-15の冷却システムは十分な冷却性能の確保を設計方針とした.昨年のエンデュランスでは水温が$110 \ {}^\circ\mathrm{C}$を超えておりこれはラジエータの配置によるものと考えられた.KS-14はラジエータの外側が後方に倒れるように搭載していたがコア部に空気がうまく流入せず空気側の熱伝達率が小さくなっていると考えられる.これに対しKS-15はコア部が正面を向くように配置しサイドポンツーンの兼ね合いで25$^\circ$前傾させた.これによって熱伝達率の向上と放熱量の向上が見込める.またラジエータを正面に向けたため従来のサイズでは車幅を超えるためサイズを縮小したところの軽量化となった.またステーの見直しにより0.43 kg軽量化できた.
