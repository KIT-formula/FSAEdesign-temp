%%%%%%%%%%%%%%%%%%%%%%%%%%%
\subsection{吸気}
%%%%%%%%%%%%%%%%%%%%%%%%%%%
昨年度の吸気系はサージタンク容量の増加とそれに伴う利用可能スペースの減少により,応答性の悪化と出力特性の低下を招いた.そこで,今年度の吸気の目標を応答性の向上と常用回転域における出力特性,特にコーナー脱出時(5000〜8000 rpm)における特性の向上とした.

%%%%%%%%%%%%%%%%%%%%%%%%%%%
\subsubsection{スロットルボディ}
%%%%%%%%%%%%%%%%%%%%%%%%%%%
スロットル径の検討について昨年までの径$\phi$25と一昨年の径$\phi$32のリストリクターを含めたモデルを三次元熱流体解析ソフトSCRYU/Tetraで定常解析を行った.スロットル開度ごとの質量流量を算出した結果,$\phi$25ではスロットル開度の増加に伴い質量流量が比較的線形で変化しているが,$\phi$32では質量流量は全体的に増加したがスロットル開度が80を超えると質量流量が減少した(Fig.\ref{fig:intake1}).また,GT-powerを用いた解析では$\phi$25と$\phi$32で出力特性に差は見られなかった.そこで,今年度は$\phi$25と$\phi$32のスロットルを製作し実走行にて評価を行った.

%%%%%%%%%%%%%%%%%%%%%%%%%%%
\subsubsection{スロットル径}
%%%%%%%%%%%%%%%%%%%%%%%%%%%
今年はGT-powerの解析結果によってスロットルの径を実走行での結果を基に決定するため,$\phi$25と$\phi$32のスロットルを製作した.これらのスロットルの形状は,ANSYS CFXでの解析結果を基に決定した.質量流量の向上を目標に設計する事により質量流量を昨年より最大で3.8 \%向上させることに成功した.

%%%%%%%%%%%%%%%%%%%%%%%%%%%
\subsubsection{出力特性}
%%%%%%%%%%%%%%%%%%%%%%%%%%%
GT-powerを用いて吸排気系でのパワー・トルク特性の評価を行った.吸排気での設計は吸気系がマシンの出力特性に及ぼす影響が大きいので吸気系の設計を先に行った.GT-powerでの設計は昨年度モデルからバルブタイミングの変更と吸気系のパラメータを変更し,主に吸気管長とタンク容量の決定を行った.ここで,吸気管長はファンネルから吸気ポートまでの距離のことを言い昨年度の130 mmから管長を長くしていった.その結果,Fig.\ref{fig:intake2}のように管長が280 mm,タンク容量3.5 Lの時にピークトルクが7000 rpmとなり5000〜8000 rpmにかけての出力特性を向上することができた.今後は実測による評価を行う予定である.

%%%%%%%%%%%%%%%%%%%%%%%%%%%
\subsubsection{サージタンク}
%%%%%%%%%%%%%%%%%%%%%%%%%%%
サージタンクの容量はGT-powerの解析による結果から3.5 Lに決定し,形状は流体解析ソフトANSYS CFXを用いて各気筒への吸気流量にばらつきがないように形状を決定した.その結果,Fig.\ref{fig:intake3}のように各気筒への吸気流量のバラつき2.5 \%以下に抑えることができた.
