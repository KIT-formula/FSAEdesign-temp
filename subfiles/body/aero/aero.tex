%%%%%%%%%%%%%%%%%%%%%%%%%%%
\section{エアロ}
%%%%%%%%%%%%%%%%%%%%%%%%%%%
\label{sec:concept}
%%問題点
KS-14はサイドポンツーンを搭載しなかった.その影響でラジエータは必要な空気の流入量を確保できず,昨年の大会では水温が$110 \ {}^\circ\mathrm{C}$にまで上昇した.また,KS-14はアンダーステアの傾向にあり,コーナー脱出直後にアクセルを踏み込むことができなかった.

%% %%改善点
%% KS-15はこれらの問題を解決するべく,KS-14の空力デバイスに加え,「サイドポンツーン」「ウイング」を新たに搭載する.これは,サイドポンツーンによる導風によってラジエータに流れ込む空気を確保し,Fr.ウイングによるDF(以下 DF)によってFr.タイヤのCFを増大させることが狙いである.また,Fr.ウイングのみを搭載すると,マシンの姿勢変化によるDFの変動によってマシンの荷重移動が顕著になり,Drivabilityの低下につながると考え,Rr.ウイングも搭載することによって前後のウイングによる空力重心を適切な位置に置き,荷重移動の抑制とDrivabilityの向上を狙う.さらに,サスペンション班からの要求として,空力デバイスによる総DF:150Nと,目標重量4.5N/kgを満たすウイングを目標とする.

%%改善点(坂井さん案)
KS-15はこれらの問題を解決するべく,KS-14の空力デバイスに加え,「サイドポンツーン」「ウイング」を新たに搭載する.これは,サイドポンツーンによる導風によってラジエータに流れ込む空気を確保し,Fr.ウイングによるDFによってFr.タイヤのCFを増大させることが狙いである.

また,車体バネ上重心とピッチ中心位置関係による車体バネ上重量の荷重移動と,姿勢変化(制動Frダイブ\&加速ピッチング)による空力重心の移動をうまくバランスさせることで,マシンのコーナリング性能の上昇を狙うためにRr.ウイングも搭載することにした.

さらに,サスペンションからの要求より,空力デバイスによる総DF:150 Nと,重量4.5 N/kgを満たすウイングを目標とする.

%%解析に関して
解析はソフトウェアクレイドル社の三次元熱流体解析ソフトSCRYU/Tetraを使用した.解析条件はアクセラレーションとスキッドパッドの状況を想定して行った.より正確な解析結果を得るために,速度は走行時の平均速度である40 km/hを基本とし,解析モデルにはヘルメットやヘッドレスト周辺のファイアウォールなどを組み込んだ.また,走行状態を再現するためにタイヤと路面には速度に応じた回転や移動条件を与え,スキッドパッドでは舵角とスリップアングルも再現した.

%%%%%%%%%%%%%%%%%%%%%%%%%%%
\subsection{Fr.ノーズ}
%%%%%%%%%%%%%%%%%%%%%%%%%%%
%% \subsection{問題と改善点}
%% %%%%%%%%%%%%%%%%%%%%%%%%%%%
%% KS-14のCFD解析結果より,ノーズ先端に高い圧力が広範囲にかかっていることがわかった.よって,高い圧力がかかる範囲を少なくするために,KS-15ではKS-14よりも先端を鋭利なものにした.CFD解析による,KS-14とKS-15のノーズにかかる表面圧力の比較をFig.○に示す.

%%%%%%%%%%%%%%%%%%%%%%%%%%%
\subsubsection{製作方法の改善}
%%%%%%%%%%%%%%%%%%%%%%%%%%%
KS-14のFr.ノーズは歪んでおり,設計通りに製作できたとは言えなかった.KS-14は断面図を貼り付けたネオマフォームに合わせて型を成型していたため,型を成型する際に目標とする形状を視認することができず,型に歪みが生じたと考えた.そこでKS-15では成型を行う前に木材の板を用いてFr.ノーズの目標とする形状を再現した.木材の切り出しはレーザーカッターを用いることで製作精度が向上した.成型を行う前に形状を再現することによって,目標とする形状を確認しながら成型することができた.この製作方法によってFr.ノーズの歪みを改善することができ,KS-14ではすべて人の手で行っていた作業を,KS-15では一部の工程にレーザーカッターを用いることで型の修正にかかる時間を短縮することができたため,製作期間を40 \%短縮することができた.

%%%%%%%%%%%%%%%%%%%%%%%%%%%
\subsection{サイドポンツーン\&サイドパネル}
%%%%%%%%%%%%%%%%%%%%%%%%%%%
CFD解析によるサイドポンツーン周辺の空気の流れをFig.\ref{fig:pontoon}に示す.

まず,KS-14の冷却性能の課題を改善するため,KS-15ではサイドポンツーンとサイドパネルを搭載した.サイドポンツーンはFr.タイヤ後方に位置するため,形状によっては回転によって生じるタイヤ後方の乱流がサイドポンツーンに流れ込み,ラジエータへの流量低下につながると考えた.したがって,サイドポンツーンの形状を,ノーズとFr.タイヤの間から来る空気を積極的に取り込むような形状とした.

また,KS-14ではコクピット内への空気の流入を防ぐために,側面下部にのみパネルを貼っていた.しかし,塞がれていない側面上部からコクピット内へ空気が流入し,空気の流れを妨げていることがCFD解析によりわかった.そこでKS-15ではこの流入を防ぐために,側面をサイドパネルで覆った.

%% %%%%%%%%%%%%%%%%%%%%%%%%%%%
%% \subsection{サイドパネル}
%% %%%%%%%%%%%%%%%%%%%%%%%%%%%
%% KS-14ではコックピット内への空気の流入を防ぐために,側面下部にのみパネルを貼っていた.しかし,塞がれていない側面上部からコックピット内へ空気が流入し,空気の流れを妨げていることがCFD解析によりわかった.そこでKS-15ではこの流入を防ぐために,側面をサイドパネルで覆った.

%%%%%%%%%%%%%%%%%%%%%%%%%%%
\subsection{ウイング}
%%%%%%%%%%%%%%%%%%%%%%%%%%%
\subsubsection{翼断面形状の決定}
%%%%%%%%%%%%%%%%%%%%%%%%%%%
%%迎角変化
KS-15の翼型はNACA-97をベースにCFD解析を行いながら設計した.NACAのレポート\cite{NACA}によると,NACA-97は迎角15$^\circ$のときにCl値が,迎角-2$^\circ$のときにCd値がそれぞれピークに達する.KS-14の平均速度40 km/hにおいて,NACA-97の迎角の変化に対する各パラメータの変化をCFD解析によって調べるとFig.\ref{fig:angle_data}のようになった.ウイングによるDFはコーナー進入時に最も必要であると考え,ブレーキング終点の姿勢変化において,KS-15で使用するウイングのメインフラップの迎角がFig.\ref{fig:angle_data}のCl値が急に落ち込む15$^\circ$となるようにした.

%%スリット
次に,NACA-97の迎角20$^\circ$のときの乱流エネルギーの分布をFig.\ref{fig:naca97_deg20}に示す.このときCl値は3.50,Cd値は0.76,L/Dは4.61であった.ウイングは迎角やキャンバーを大きくしていくことで,ウイング裏面の反り返りが強くなり,空気の剥離が起こる.空気の剥離が起こることによってウイング後方には負圧が発生し,それにより生じる圧力抗力が空気抵抗となる.迎角20$^\circ$のときのNACA-97にも同様の現象が見られたため,空気の剥離点付近にスリットを設けることでウイング前方から後方へ空気を送りこみ,空気の剥離を抑えた.迎角20$^\circ$の一枚翼にスリットを設けて三分割した翼断面の乱流エネルギーの分布をFig.\ref{fig:naca97_151515}に示す.CFD解析の結果より,スリットを設けることで空気の剥離は改善され,一枚翼と比べてCl値は5.26となり50 \%,Cd値は1.03となり0.36 \%上昇した.L/Dは5.11となり,0.11 \%上昇した.

以上の解析で得られたデータを用いてウイングの翼断面の形状を決定した.

%%%%%%%%%%%%%%%%%%%%%%%%%%%
\subsubsection{Fr.ウイング}
%%%%%%%%%%%%%%%%%%%%%%%%%%%
%%立ち位置的な
Fr.ウイングは%マシンの中で最初に風に接触する部分であり,Fr.ウイングから%
後方のパーツの空気の流れに影響を与える重要なパーツである.また,地面に最も近い位置に存在するため,グランドエフェクトによるDF増加も期待できると考える.

%%形状
KS-15のFr.ウイングの概形はサイドポンツーンに流入する風の向きを妨げないように,ノーズ付近のフラップを低くした.

翼端板はタイヤ後方に生じる乱流を減らす目的で外側へ反らす形状とすることで空気がタイヤに衝突しづらくし,かつフラップ下方の空気の流速を速めることを狙った,さらにフラップ下方の負圧に向かって外から空気が流れこむのを防ぐためにフットフランジを設けた.

翼端板と同様の目的で,コーナリング時に空気がタイヤに衝突しづらくなるように,フラップ下方にスプリッターを設けた.

%%また,タイヤ直前に位置するフラップによってはね上げられた空気はタイヤ上方に衝突する.この衝突によって,タイヤ後方に乱流が生じ

%% 風の向きとタイヤ表面の進行方向が逆であるため,摩擦抗力が生じ,摩擦によってタイヤ温度の過度な上昇につながることが考えられる.そこで,フラップの両端に壁を設けて空気の流れを制限するのではなく,フラップ上面を流れる空気を横へ逃がし,タイヤに衝突する空気の量を減らす目的でフラップの両端のうち,内側の壁を除いた.CFD解析による,壁を設けた場合と除いた場合のタイヤの表面圧力の比較をFig.\ref{fig:tire_acc}とFig.\ref{fig:tire_skid}に示す.

%% タイヤ正面の正圧によって空気の剥離が起きやすくなると考え,空気の剥離を抑えるために,Fr.ウイングの翼端板の形状をウイングの外から空気を送り込むようにした.%%よく分かってないけどね

%%最低地上高
Fr.ウイングの最低地上高はグランドエフェクトによるDFを得るために,ボディの最低地上高よりも5 mmの余裕を持って35 mmとした.

%%%%%%%%%%%%%%%%%%%%%%%%%%%
\subsubsection{Rr.ウイング}
%%%%%%%%%%%%%%%%%%%%%%%%%%%
%%立ち位置的な
Rr.ウイング前方にはメインフープ周辺のファイアウォールや吸気系パーツなど,空気の流れを妨げるものが多く存在するため,Rr.ウイングはそれらによって生じる乱流や負圧の影響を受けやすい.また,Rr.ウイングはFr.ウイングに比べて大規模なものにできるため,大きなDFを得やすい反面,前方投影面積の増大によるドラッグの増加も起こりやすいと考えられる.

%%マウント方法
バネ上マウントはマシンの姿勢変化によってRr.ウイングの迎角が変化すると考えた.したがってKS-15ではマシンの姿勢変化に影響を受けず,またサスペンションを経由することなく,直接タイヤにDFを印加できるバネ下マウントを採用した.

%%高さ
Rr.ウイング前方のパーツによって生じる乱流や負圧の影響を回避するために,Rr.ウイングは前方のパーツから離れた位置に配置したい.しかし,慣性モーメントは距離の二乗に比例するため,重心から離れた位置に配置すると,慣性モーメントの増加によってコーナリング性能に影響が及ぶ.KS-15のRr.ウイングはメインフープよりも低く設定し,バネ下マウントであることを考慮した上で可能な限り後方へ配置した.

%%%%%%%%%%%%%%%%%%%%%%%%%%%
\subsubsection{空力重心}
%%%%%%%%%%%%%%%%%%%%%%%%%%%
前後ウイングによって生じるモーメントがつり合う点を空力重心とし,この点がサスペンション班が定める前後重量比に近づくようにウイングを配置した.導出にはCFD解析によって得た前後ウイングそれぞれにかかるDFと空気による圧力モーメントを用いた.レギュレーションの範囲内でウイングの位置を調整し,空力重心を50:50とした.

%%%%%%%%%%%%%%%%%%%%%%%%%%%
\subsection{製作・解析精度の検証}
%%%%%%%%%%%%%%%%%%%%%%%%%%%
\label{sec:experiment}
製作精度の影響により,CFD解析と同じような結果が得られないことも考えられるため,製作・解析精度の検証を行った.検証はタフト法による流れの可視化を行った.

%と,センサによるDFと解析によるDFの比較を行った.

%%間に合えば書く.無理ならデザインパネルで話す
タフト法はRr.ウイングのフラップ下面に対して行った.検証結果をFig.\ref{fig:taft_acc},Fig.\ref{fig:taft_skid}に示す.

%ストロークセンサをFr.サスペンションに,歪みゲージをRr.ウイングのステーにそれぞれ取り付けて計測した.検証結果をFig.◯に示す.

%%%%%%%%%%%%%%%%%%%%%%%%%%%
\subsection{エアロ開発まとめ}
%%%%%%%%%%%%%%%%%%%%%%%%%%%
KS-14での問題点を改善するために,KS-15では新たな空力デバイスを追加した.デバイス単体でのCFD解析(@40 km/h)を行ったところ,Fr.ウイングはDF:62 N,L/D:4.7となり,Rr.ウイングはDF:95 N,L/D:3.8となることがわかった.しかし,空力デバイスをボディに搭載したモデルで解析をかけたところ,メインフープ後方にあるRr.ウイングがDF:85 N,L/D:2.9となり,単体での解析結果と比べてDFが11 \%減,L/Dが26 \%減となった.

\ref{sec:experiment}項で行ったタフト法の検証結果より,KS-15に搭載したRr.ウイングがメインフープ周辺パーツの影響を受け,本来の性能を引き出せていないのは明白である.ここで,ウイング単体での解析結果をサスペンションからの要求性能4.5 N/kgを適用すると,ウイングの目標重量はFr.ウイングは13 kg,Rr.ウイングは21 kgとなる.実際に制作したウイングはFr.ウイングは5 kg,Rr.ウイングは6.5 kgであるため,これらは\ref{sec:concept}項で述べたサスペンションの要求を達成している.

以上より,今後はマシン全体の流れに注目し,空力デバイスの性能を引き出すために「流れの一貫性」を確立することが課題になると考える.
